% LaTeX file for resume

% possible options include:
%   font size (10pt, 11pt and 12pt)
%   paper size (a4paper, a5paper, letterpaper, legalpaper, and landscape)
%   font family (sans and roman)
\documentclass[10pt,letterpaper,sans]{moderncv}

%\usepackage{fontspec}

\usepackage{tikz}

\newcommand{\ghCommit}[1]{%
\begin{tikzpicture}[y=0.7pt,x=0.7pt,yscale=-1, inner sep=0pt, outer sep=0pt]%
  \path[even odd rule,fill=#1]%
  (10.86,7.00) .. controls (10.41,5.28) and%
  (8.86,4.00) .. (7.00,4.00) .. controls (5.14,4.00) and%
  (3.59,5.28) .. (3.14,7.00) -- (0.00,7.00) -- (0.00,9.00) --%
  (3.14,9.00) .. controls (3.59,10.72) and (5.14,12.00) ..%
  (7.00,12.00) .. controls (8.86,12.00) and (10.41,10.72) ..%
  (10.86,9.00) -- (14.00,9.00) -- (14.00,7.00) -- (10.86,7.00)%
  -- cycle(7.00,10.20) .. controls (5.78,10.20) and (4.80,9.22) ..%
  (4.80,8.00) .. controls (4.80,6.78) and (5.78,5.80) ..%
  (7.00,5.80) .. controls (8.22,5.80) and (9.20,6.78) ..%
  (9.20,8.00) .. controls (9.20,9.22) and (8.22,10.20) ..%
  (7.00,10.20) -- cycle;%
\end{tikzpicture}}

% style options are casual (default), classic, banking, oldstyle and fancy
% color options black, blue (default), burgundy, green, grey, orange, and red
\moderncvstyle{casual}
\moderncvcolor{blue}
%\moderncvicons{marvosym}

% to set the default font:
%   use \sfdefault for the default sans serif font
%   use \rmdefault for the default roman one
%   or use any TeX font name
\renewcommand{\familydefault}{\sfdefault}

% Disable page numbers
\nopagenumbers{}

% character encoding
%\usepackage[utf8]{inputenc}

% adjust the page margins
\usepackage[scale=0.75]{geometry}

% extra packages
\usepackage{booktabs}

% patch moderncv's \cventry to not include a '.' at the end of its args
% see https://tex.stackexchange.com/questions/241093/moderncv-cventry-remove-dots-at-the-end-of-line
\usepackage{xpatch}
\xpatchcmd{\cventry}{.\strut}{\strut}{}{}

% if you want to change the width of the column with the dates
%%\setlength{\hintscolumnwidth}{3cm}

% if you want to force the width allocated to your name and avoid line breaks
%%\setlength{\makecvtitlenamewidth}{10cm}

\name{Daniel}{Cartwright}
\email{chessai1996@gmail.com}
%\homepage{chessai.me} # nothing there, currently
\social[github]{chessai}
\social[linkedin]{chessai}

% TODO: make this open in new tab, somehow
\newcommand{\wlink}[2]{\textcolor[HTML]{461645}{\href{#1}{#2}}}

\newcommand{\nixpkg}[2]{%
  \wlink{https://github.com/NixOS/nixpkgs/tree/master/pkgs/#1/#2/default.nix}%
        {#2}%
}

\newcommand{\ghlink}[2]{\wlink{https://github.com/#1}{#2}}
\newcommand{\ghrepo}[1]{\ghlink{#1}{\faGithub}}
\newcommand{\ghlang}[1]{\texttt{#1}}
\newcommand{\ghcom}[1]{\textcolor[HTML]{666666}{\ghCommit{} #1}}
\newcommand{\ghadd}[1]{\textcolor[HTML]{30A622}{{\faPlusCircle} #1}}
\newcommand{\ghrem}[1]{\textcolor[HTML]{BD2C00}{{\faMinusCircle} #1}}
\newcommand{\ghstar}[1]{#1}
\newcommand{\ghfork}[1]{#1}
\newcommand{\ghtr}[0]{}
\newcommand{\ghtf}[0]{\faCodeFork}

\newcommand{\ghub}[4]{\ghrepo{#2} | \ghlang{#1} | \ghadd{#3} | \ghrem{#4}}
\newcommand{\ghtable}[6]{#1 & #2 & #3 & #4 & #5 & #6 \\}
\newcommand{\ght}[9]{%
  \ghtable{#1}
          {\ghlink{#2/#3}{#3}}
          {\ghlang{#4}}
          {\ghcom{#5}}
          {\ghadd{#6}}
          {\ghrem{#7}}%
}

\newcommand{\lang}[1]{\texttt{#1}}
%\newcommand{\langskill}[2]{\href{https://en.wikipedia.org/wiki/#1_(programming_language)}{\lang{#2}}}
%\newcommand{\lstable}[3]{\langskill{#1}{#2} & #3 \\}

\newcommand{\course}[2]{\texttt{#1 #2}}
\newcommand{\credit}[0]{$\star$}

\usepackage{tabularx}

\begin{document}
\makecvtitle{}

%\vspace{0.6em}

\section{Work Experience}
\cventry{2020--Present}
  {Software Engineer}
  {\wlink{https://facebook.com}{Facebook}}
  {}
  {}
  {
\begin{itemize}
  \item I maintain \wlink{https://github.com/facebook.com/duckling}{Duckling}, an open source library for parsing text into structured data
  \item I co-maintain \wlink{https://wit.ai}{wit.ai}, a solution for providing third-party users access to Facebook's NLP capabilities
\end{itemize}
}

\cventry{2020--2021}
  {Haskell Programmer}
  {\wlink{https://mercury.com}{Mercury, Banking for Startups}}
  {}
  {}
  {
\begin{itemize}
  \item I worked on a banking webserver using \lang{Haskell} and \wlink{https://www.yesodweb.com/}{Yesod}.
  \item I have done some small amounts of frontend development for \wlink{https://mercury.com}{mercury.com}, which is written with \lang{TypeScript} + \lang{React}.
  \item I co-maintained a lot of the infrastructure for \textcolor{blue}{\wlink{https://mercury.com}{Mercury}}, which at the time was comprised of: \wlink{https://aws.amazon.com/}{AWS}, \wlink{https://nixos.org/}{\lang{Nix}}, \wlink{https://nixos.wiki/wiki/Hydra}{Hydra}, \wlink{https://www.terraform.io/}{Terraform}, and \wlink{https://dhall-lang.org/}{\lang{Dhall}}.
\end{itemize}
}

\cventry{2017--2019}
  {Haskell Programmer}
  {\wlink{https://layer3com.com}{Layer 3 Communications, LLC.}}
  {}
  %\newline I develop and maintain a suite of network security tools in \lang{Haskell} as part of a small team.
  {}
  {
\begin{itemize}
  \item I developed and maintained a suite of network security tools in \lang{Haskell} as part of a small team.
  \begin{itemize}
    \item \textcolor{blue}{Allsight} - A distributed \wlink{https://en.wikipedia.org/wiki/Security_information_and_event_management}{SIEM}. The tool ingests and analyses syslog, and from this analysis it uses rules defined by security experts to detect both single-log and multi-log (correlated) events, on which it alerts. There is a GUI for our security team to configure rules and view collected data. Clients can also use the GUI to view data relevant to them.
    \item \textcolor{blue}{Diamond} - A network performance monitoring system. Uses \wlink{https://en.wikipedia.org/wiki/Simple_Network_Management_Protocol}{SNMP} to gather metrics from network devices (e.g. interface throughput; utilization of CPU, memory, storage, power). The tool is fully concurrent; thousands of hosts can be polled in about 30 seconds total. These metrics are normalized and pushed into \wlink{https://kafka.apache.org/}{Apache Kafka}. The data is tracked by an alerting tool and sent to \wlink{https://www.influxdata.com/}{InfluxDB}/\wlink{https://grafana.com/}{Grafana}.
    \item \textcolor{blue}{Netcrawl} - Uses \wlink{https://en.wikipedia.org/wiki/Simple_Network_Management_Protocol}{SNMP} and \wlink{https://en.wikipedia.org/wiki/Link_Layer_Discovery_Protocol}{LLDP} to brute-force the discovery of a network, given only a subnet or set of subnets. The tool collects a variety of useful data about each node in the network, and outputs a summary which can be analysed by human or another tool. The graph of the network can be output as a \wlink{http://www.graphviz.org/}{GraphViz} dot file.
  \end{itemize}
\end{itemize}
}

%\vspace{0.6em}

\section{Open Source Programming}
\cventry{2017--Present}
  {Maintainer \& Contributor}
  {\wlink{https://github.com/chessai}{chessai}}
  {\newline I began writing \lang{Haskell} in August of 2017, \lang{Nix} shortly after. Since then, I have contributed to over 200 open source \lang{Haskell} projects. I actively maintain or co-maintain roughly 100 open source \lang{Haskell} libraries. I am a member of the \wlink{https://wiki.haskell.org/Core_Libraries_Committee}{Haskell Core Libraries Committe}, which oversees and maintains the core libraries that make up the \lang{Haskell} ecosystem. I am the chief maintainer of the \lang{Haskell} standard library, base. I am a drive-by contributor of the \wlink{https://www.haskell.org/ghc/}{Glasgow Haskell Compiler}. Listed are just a few projects to which I contribute proudly.
  }
  {}
  {
\begin{itemize}
  \item duckling: Haskell library for parsing text into structured data. \newline \ghub{Haskell}{facebook/duckling}{3738}{2275}
  \item refined: Embedding simple refinement types inside of GHC Haskell. Supports run-time and compile-time refinements. \newline \ghub{Haskell}{chessai/refined}{7687}{6035}
  \item nixpkgs: the nixpkgs repo. \newline \ghub{Nix}{NixOS/nixpkgs}{2807}{36}
  \item nixos-configs: My NixOS configs. \newline \ghub{Nix}{chessai/nixos-configs}{5965}{3574}
\end{itemize}
}

\end{document}
