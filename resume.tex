% LaTeX file for resume

% possible options include:
%   font size (10pt, 11pt and 12pt)
%   paper size (a4paper, a5paper, letterpaper, legalpaper, and landscape)
%   font family (sans and roman)
\documentclass[10pt,letterpaper,sans]{moderncv}

%\usepackage{fontspec}

\usepackage{tikz}

\newcommand{\ghCommit}[1]{%
\begin{tikzpicture}[y=0.7pt,x=0.7pt,yscale=-1, inner sep=0pt, outer sep=0pt]%
  \path[even odd rule,fill=#1]%
  (10.86,7.00) .. controls (10.41,5.28) and%
  (8.86,4.00) .. (7.00,4.00) .. controls (5.14,4.00) and%
  (3.59,5.28) .. (3.14,7.00) -- (0.00,7.00) -- (0.00,9.00) --%
  (3.14,9.00) .. controls (3.59,10.72) and (5.14,12.00) ..%
  (7.00,12.00) .. controls (8.86,12.00) and (10.41,10.72) ..%
  (10.86,9.00) -- (14.00,9.00) -- (14.00,7.00) -- (10.86,7.00)%
  -- cycle(7.00,10.20) .. controls (5.78,10.20) and (4.80,9.22) ..%
  (4.80,8.00) .. controls (4.80,6.78) and (5.78,5.80) ..%
  (7.00,5.80) .. controls (8.22,5.80) and (9.20,6.78) ..%
  (9.20,8.00) .. controls (9.20,9.22) and (8.22,10.20) ..%
  (7.00,10.20) -- cycle;%
\end{tikzpicture}}

% style options are casual (default), classic, banking, oldstyle and fancy
% color options black, blue (default), burgundy, green, grey, orange, and red
\moderncvstyle{casual}
\moderncvcolor{blue}
%\moderncvicons{marvosym}

% to set the default font:
%   use \sfdefault for the default sans serif font
%   use \rmdefault for the default roman one
%   or use any TeX font name
\renewcommand{\familydefault}{\sfdefault}

% Disable page numbers
\nopagenumbers{}

% character encoding
%\usepackage[utf8]{inputenc}

% adjust the page margins
\usepackage[scale=0.75]{geometry}

% extra packages
\usepackage{booktabs}

% if you want to change the width of the column with the dates
%%\setlength{\hintscolumnwidth}{3cm}

% if you want to force the width allocated to your name and avoid line breaks
%%\setlength{\makecvtitlenamewidth}{10cm}

\name{Daniel}{Cartwright}
\email{chessai1996@gmail.com}
\homepage{chessai.me}
\social[github]{chessai}
\social[linkedin]{chessai}

\newcommand{\wlink}[2]{\textcolor[HTML]{461645}{\href{#1}{#2}}}

\newcommand{\nixpkg}[2]{%
  \wlink{https://github.com/NixOS/nixpkgs/tree/master/pkgs/#1/#2/default.nix}%
        {#2}%
}

\newcommand{\ghlink}[2]{\wlink{https://github.com/#1}{#2}}
\newcommand{\ghrepo}[1]{\ghlink{#1}{\faGithub}}
\newcommand{\ghlang}[1]{\texttt{#1}}
\newcommand{\ghcom}[1]{\textcolor[HTML]{666666}{\ghCommit{} #1}}
\newcommand{\ghadd}[1]{\textcolor[HTML]{30A622}{{\faPlusCircle} #1}}
\newcommand{\ghrem}[1]{\textcolor[HTML]{BD2C00}{{\faMinusCircle} #1}}
\newcommand{\ghstar}[1]{#1}
\newcommand{\ghfork}[1]{#1}
\newcommand{\ghtr}[0]{}
\newcommand{\ghtf}[0]{\faCodeFork}

\newcommand{\ghub}[4]{\ghrepo{#2} | \ghlang{#1} | \ghadd{#3} | \ghrem{#4}}
\newcommand{\ghtable}[6]{#1 & #2 & #3 & #4 & #5 & #6 \\}
\newcommand{\ght}[9]{%
  \ghtable{#1}
          {\ghlink{#2/#3}{#3}}
          {\ghlang{#4}}
          {\ghcom{#5}}
          {\ghadd{#6}}
          {\ghrem{#7}}%
}

\newcommand{\lang}[1]{\texttt{#1}}

\newcommand{\course}[2]{\texttt{#1 #2}}
\newcommand{\credit}[0]{$\star$}

\usepackage{tabularx}

\begin{document}
\makecvtitle{}

%\vspace{0.6em}

\section{Work Experience}
\cventry{2017--Present}
        {Programmer}
        {\wlink{https://www.layer3com.com}{Layer 3 Communications, LLC.}}
        {}{}
        {
\begin{itemize}
\item I am currently working on a network \wlink{https://en.wikipedia.org/wiki/Security_information_and_event_management}{SIEM} using \lang{Haskell}. It requires managing a distributed system that ingests network logs, enriching them with metadata, and matching the enriched logs against a set of rules, that when fired, create an incident which alerts any relevant security personnel. A correlator service takes the enriched logs and correlates them, so that "correlated rules" can fire, i.e., multi-event incidents can exist. The SIEM architecture will be moving to utilise \wlink{https://kafka.apache.org/}{Apache Kafka} instead of \wlink{https://www.rsyslog.com/}{rsyslog}.
\item I work on a suite of network performance monitoring (NPM) tools written in \lang{Haskell}. Among the tasks it performs are polling devices with SNMP for things such as CPU, memory, or strorage utilisation, interface throughput, and using a custom \wlink{https://github.com/andrewthad/ping}{pinging utility implemented in \lang{Haskell}} to report up/down statistics of various devices. All collected metrics are pushed into \wlink{https://kafka.apache.org/}{Apache Kafka}, while a consumer listens and pushes metrics to \wlink{https://graphiteapp.org/}{Carbon}, where clients can view the data as a graph.
\end{itemize}
}

%\vspace{0.6em}

\section{Open Source Programming}
\cventry{}
        {\wlink{https://github.com/chessai}{chessai}}
        {}{}{}
        {
I began writing \lang{Haskell} in August of 2018. Since then, I have contributed to over 200 open source \lang{Haskell} projects, and I currently maintain over 50 open source \lang{Haskell} libraries. Below are a few libraries to which I contribute proudly.
}

\cventry{}
        {\wlink{https://github.com/andrewthad/quickcheck-classes}{quickcheck-classes}}
        {}{}{}
        {
\begin{itemize}
\item A library wrapping QuickCheck that provides the ability to rapidly test typeclass laws of common typeclasses (such as all lawful typeclasses in base, or things like the partial isomporphism constituted by toJSON/fromJSON), but designed with API simplicity in mind. The motto of the library is roughly "if a user knows how to use QuickCheck, they will know quickly how to use quickcheck-classes". This library has been used to catch bugs in multiple of my own/coworkers libraries/executables, including libraries such as \wlink{https://github.com/haskell/primitive}{primitive}, where it caught a number of bugs. A WIP port of the library that is based on hedgehog is available \wlink{https://github.com/chessai/hedgehog-classes}{here}.
\item \ghub{Haskell}{andrewthad/quickcheck-classes}{}{}
\end{itemize}
}

%\cventry{}
%        {\wlink{https://github.com/nikita-volkov/refined}{refined}}
%        {}{}{}
%        {
%\begin{itemize}
%\item Simple refinement types inside of GHC Haskell, with a straightforward, lightweight API.
%\item \ghub{Haskell}{nikita-volkov/refined}{}{}
%\end{itemize}
%}

\cventry{}
        {\wlink{https://github.com/haskell-streaming/streaming}{streaming}}
        {}{}{}
        {
\begin{itemize}
\item A streaming API for \lang{Haskell}. I both maintain and use this library, as well as related libraries such as \wlink{https://github.com/haskell-streaming/streaming-bytestring}{streaming-bytestring}.
\item \ghub{Haskell}{haskell-streaming/streaming}{}{}
\end{itemize}
}

\cventry{}
        {\wlink{https://github.com/chessai/semirings}{semirings}}
        {}{}{}
        {
\begin{itemize}
\item A straightforward API for \wlink{https://en.wikipedia.org/wiki/Semiring}{semirings}. Semirings are a useful structure that have numerous applications in mathematics and computer science. For example, the \wlink{https://en.wikipedia.org/wiki/Deterministic_finite_automaton}{DFA} and \wlink{https://en.wikipedia.org/wiki/Nondeterministic_finite_automaton}{NFA} are semirings, with useful observations being derived from their instances (see \wlink{https://github.com/andrewthad/automata/blob/c7936f3705bf9b330c42f62d0350037ecad611dd/src/Automata/Internal.hs##L427-L439}{here}). Other useful applications of semirings and star-semirings can be found at \wlink{http://r6.ca/blog/20110808T035622Z.html}{this blog post} by Ross Patterson.
\item \ghub{Haskell}{chessai/semirings}{}{}
\end{itemize}
}

\section{Skills}
\cvitem{Programming}{%
  \href{https://en.wikipedia.org/wiki/Haskell_(programming_language)}{\lang{Haskell}}, %
  \href{https://en.wikipedia.org/wiki/Nix_package_manager}{\lang{Nix}} %
  % \href{https://en.wikipedia.org/wiki/Agda_(programming_language)}{\lang{Agda}}, %
  % \href{https://en.wikipedia.org/wiki/Rust_(programming_language)}{\lang{Rust}}, %
  % \href{https://en.wikipedia.org/wiki/C_(programming_language)}{\lang{C}}, %
  % \href{https://en.wikipedia.org/wiki/C++}{\lang{C++}}, %
  % \href{https://en.wikipedia.org/wiki/Java_(programming_language)}{\lang{Java}}, %
  % \href{https://en.wikipedia.org/wiki/OCaml}{\lang{OCaml}}, %
  % \href{https://en.wikipedia.org/wiki/Python_(programming_language)}{\lang{Python}}%
}
\cvitem{Software}{%
  \href{https://en.wikipedia.org/wiki/Linux}{Linux}, %
  \href{https://nixos.org}{NixOS}, %
  \href{https://git-scm.com}{git}, %
  \href{https://www.latex-project.org}{\LaTeX} %
}

\end{document}
