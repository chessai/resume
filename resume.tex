% LaTeX file for resume

% possible options include:
%   font size (10pt, 11pt and 12pt)
%   paper size (a4paper, a5paper, letterpaper, legalpaper, and landscape)
%   font family (sans and roman)
\documentclass[10pt,letterpaper,sans]{moderncv}

%\usepackage{fontspec}

\usepackage{tikz}

\newcommand{\ghCommit}[1]{%
\begin{tikzpicture}[y=0.7pt,x=0.7pt,yscale=-1, inner sep=0pt, outer sep=0pt]%
  \path[even odd rule,fill=#1]%
  (10.86,7.00) .. controls (10.41,5.28) and%
  (8.86,4.00) .. (7.00,4.00) .. controls (5.14,4.00) and%
  (3.59,5.28) .. (3.14,7.00) -- (0.00,7.00) -- (0.00,9.00) --%
  (3.14,9.00) .. controls (3.59,10.72) and (5.14,12.00) ..%
  (7.00,12.00) .. controls (8.86,12.00) and (10.41,10.72) ..%
  (10.86,9.00) -- (14.00,9.00) -- (14.00,7.00) -- (10.86,7.00)%
  -- cycle(7.00,10.20) .. controls (5.78,10.20) and (4.80,9.22) ..%
  (4.80,8.00) .. controls (4.80,6.78) and (5.78,5.80) ..%
  (7.00,5.80) .. controls (8.22,5.80) and (9.20,6.78) ..%
  (9.20,8.00) .. controls (9.20,9.22) and (8.22,10.20) ..%
  (7.00,10.20) -- cycle;%
\end{tikzpicture}}

% style options are casual (default), classic, banking, oldstyle and fancy
% color options black, blue (default), burgundy, green, grey, orange, and red
\moderncvstyle{casual}
\moderncvcolor{blue}
%\moderncvicons{marvosym}

% to set the default font:
%   use \sfdefault for the default sans serif font
%   use \rmdefault for the default roman one
%   or use any TeX font name
\renewcommand{\familydefault}{\sfdefault}

% Disable page numbers
\nopagenumbers{}

% character encoding
%\usepackage[utf8]{inputenc}

% adjust the page margins
\usepackage[scale=0.75]{geometry}

% extra packages
\usepackage{booktabs}

% if you want to change the width of the column with the dates
%%\setlength{\hintscolumnwidth}{3cm}

% if you want to force the width allocated to your name and avoid line breaks
%%\setlength{\makecvtitlenamewidth}{10cm}

\name{Daniel}{Cartwright}
\email{chessai1996@gmail.com}
\homepage{chessai.me}
\social[github]{chessai}
\social[linkedin]{chessai}

\newcommand{\wlink}[2]{\textcolor[HTML]{0020B6}{\href{#1}{#2}}}

\newcommand{\nixpkg}[2]{%
  \wlink{https://github.com/NixOS/nixpkgs/tree/master/pkgs/#1/#2/default.nix}%
        {#2}%
}

\newcommand{\ghlink}[2]{\wlink{https://github.com/#1}{#2}}
\newcommand{\ghrepo}[1]{\ghlink{#1}{\faGithub}}
\newcommand{\ghlang}[1]{\texttt{#1}}
\newcommand{\ghcom}[1]{\textcolor[HTML]{666666}{\ghCommit{} #1}}
\newcommand{\ghadd}[1]{\textcolor[HTML]{30A622}{{\faPlusCircle} #1}}
\newcommand{\ghrem}[1]{\textcolor[HTML]{BD2C00}{{\faMinusCircle} #1}}
\newcommand{\ghstar}[1]{#1}
\newcommand{\ghfork}[1]{#1}
\newcommand{\ghtr}[0]{}
\newcommand{\ghtf}[0]{\faCodeFork}

\newcommand{\ghub}[4]{\ghrepo{#2} | \ghlang{#1} | \ghadd{#3} | \ghrem{#4}}
\newcommand{\ghtable}[6]{#1 & #2 & #3 & #4 & #5 & #6 \\}
\newcommand{\ght}[9]{%
  \ghtable{#1}
          {\ghlink{#2/#3}{#3}}
          {\ghlang{#4}}
          {\ghcom{#5}}
          {\ghadd{#6}}
          {\ghrem{#7}}%
}

\newcommand{\lang}[1]{\texttt{#1}}

\newcommand{\course}[2]{\texttt{#1 #2}}
\newcommand{\credit}[0]{$\star$}

\usepackage{tabularx}

\begin{document}
\makecvtitle{}

%\vspace{0.6em}

\section{Work Experience}
\cventry{2017--Present}
        {Programmer}
        {\wlink{https://www.layer3com.com}{Layer 3 Communications}}
        {}{}
        {
\begin{itemize}
  \item Worked on a network \wlink{https://en.wikipedia.org/wiki/Security_information_and_event_management}{SIEM} using \lang{Haskell}.
\item Setup and maintained a \wlink{https://github.com/NixOS/hydra}{Hydra server} for use as GitHub \wlink{https://en.wikipedia.org/wiki/Continuous_integration}{CI}. The server was capable of distributing pre-built binaries to developers.
\end{itemize}
}

%\vspace{0.6em}

\section{Activities}
\cventry{2016--}
        {Hobby}
        {\wlink{https://github.com/chessai}{Open Source Programming}}
        {}{}{
\setlength{\tabcolsep}{8pt}
\begin{tabular}{l l l l l l}
  \midrule
  \ght{\ghtr{}}{chessai}{semirings}{Haskell}{44}{469}{157}{0}{4}
  % Library providing a 'Semiring' typeclass, along with instances for all relevant types in \lang{Haskell}'s base library, as well as some generally useful Semiring newtypes.
  \ght{\ghtr{}}{chessai}{diet}{Haskell}{21}{3577}{2815}{0}{6}
  % First complete design and implementation of Discrete Interval Encoding Trees, a class of data structure for efficient and compact storage/lookup of enumerable data 
  \ght{\ghtr{}}{chessai}{silvi}{Haskell}{91}{5052}{4227}{0}{3}
  % Generate fake data for testing
  \ght{\ghtr{}}{chessai}{freq}{Haskell}{20}{358501}{17161}{0}{0}
  % Cryptanalytic frequency analysis tool, using a linguistic n-gram approach. Used at Layer 3 Communications to score domain names by validity. 
  \midrule
\end{tabular}
}

\cventry{2010--2014}
        {Club}
        {\wlink{http://www.uiltexas.org/academics/stem}{University Interscholastic League - STEM}}
        {Member}{}{
        Academic competitions organised by the University of Texas at Austin. \\ %
        Participated in %
        \wlink{http://www.uiltexas.org/academics/stem/mathematics}{Mathematics}, %
        \wlink{http://www.uiltexas.org/academics/stem/computer-science}{Computer Science}, \\ %
        \wlink{http://www.uiltexas.org/academics/stem/calculator-applications}{Calculator Applications}, %
        \wlink{http://www.uiltexas.org/academics/stem/number-sense}{Number Sense}, %
        \wlink{http://www.uiltexas.org/academics/stem/science}{Science}, and %
        Placed State in 2011, Regionals in 2012, 2013, and 2014.
}          

\section{Skills}
\cvitem{Programming}{%
  \href{https://en.wikipedia.org/wiki/Haskell_(programming_language)}{\lang{Haskell}}, %
  \href{https://en.wikipedia.org/wiki/Nix_package_manager}{\lang{Nix}}, %
  % \href{https://en.wikipedia.org/wiki/Agda_(programming_language)}{\lang{Agda}}, %
  % \href{https://en.wikipedia.org/wiki/Rust_(programming_language)}{\lang{Rust}}, %
  % \href{https://en.wikipedia.org/wiki/C_(programming_language)}{\lang{C}}, %
  % \href{https://en.wikipedia.org/wiki/C++}{\lang{C++}}, %
  % \href{https://en.wikipedia.org/wiki/Java_(programming_language)}{\lang{Java}}, %
  % \href{https://en.wikipedia.org/wiki/OCaml}{\lang{OCaml}}, %
  % \href{https://en.wikipedia.org/wiki/Python_(programming_language)}{\lang{Python}}%
}
\cvitem{Software}{%
  \href{https://en.wikipedia.org/wiki/Linux}{Linux}, %
  \href{https://nixos.org}{NixOS}, %
  \href{https://git-scm.com}{git}, %
  \href{https://www.latex-project.org}{\LaTeX}, %
}

\end{document}
